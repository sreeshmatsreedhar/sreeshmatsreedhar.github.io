%%%%%%%%%%%%%%%%%%%%%%%%%%%%%%%%%%%%%%%%%%%%%%%%%%%%%%%%
% Author: Sreeshma T                                %
%                                          %
%%%%%%%%%%%%%%%%%%%%%%%%%%%%%%%%%%%%%%%%%%%%%%%%%%%%%%%%

% -------------- Resume ---------------
\documentclass{resume}
\usepackage{keycommand}

\begin{document}

% --------- Contact Information -----------
\introduction[
    fullname={SREESHMA T},
    email={sreeshmatsreedhar@gmail.com},
    phone={+91 7594990969},
    linkedin={linkedin.com/in/sreeshma-t-4837472b7/},
    github={github.com/sreeshmatsreedhar}
]

% --------- Summary -----------
\summary{M.Sc. Applied Physics student specializing in VLSI Design, with internship experience at ProvLogic focused on digital design and verification. Project work includes RTL implementation of a 32-bit advanced high-speed serial CRC. }



% --------- Skills -----------
\begin{skillsSection}{Technical Skills}
    \skillItem[
        category={Programming Languages},
        skills={C++, Python, Embedded C}
    ] 
    \skillItem[
        category={Tools},
        skills={Proteus Design Suite, LTspice, Arduino IDE}
    ] \\
    \skillItem[
        category={Microcontrollers},
        skills={ESP 8266, ATMEGA382P}
    ] 
    \skillItem[
        category={FPGA Platform},
        skills={Basys 3}
    ]
    
\end{skillsSection}

% --------- Experience -----------
\begin{experienceSection}{Professional Experience}
    \experienceItem[
        company={ProV Logic},
        location={(Online Internship)},
        position={Intern},
        duration={Feb 2025 – Present}
    ]
   \begin{itemize}
    \item \textbf{32-bit Advanced High-Speed Serial CRC Design for Real-Time Applications (EDA Playground)} 
    \item Implemented a 32-bit CRC generator based on the IEEE 802.3 standard using Verilog HDL. Designed the architecture using shift registers, XOR gates, and a Linear Feedback Shift Register (LFSR).
\end{itemize}
    \experienceItem[
        company={India Innovation Center for Graphene (IICG)},
        location={Kerala University of Digital Sciences},
        position={Intern},
        duration={Feb 2024 – Jun 2024}
    ]
    \begin{itemize}
        \itemsep -6pt {}
        \item \textbf{Mxene Based Electrochemical Smart Window}: A simulation study for enhancing building energy efficiency (COMSOL Multiphysics)
        \item Simulated electrochromic performance of \(\mathrm{Ti}_3\mathrm{C}_2\mathrm{Tx}\) MXene-based smart windows in two configurations.
    \end{itemize}

\end{experienceSection}

% --------- Projects -----------
\begin{experienceSection}{Academic projects}
\projectItem[
        title={Home Automation with Blynk using NodeMCU ESP8266},
        duration={Sept 2023 – Jan 2024},
        keyHighlight={Tools \& Technologies: NodeMCU (ESP8266), Arduino IDE, Blynk App}
    ]
     \begin{itemize}
        \vspace{-0.5em}
        \itemsep -6pt {}
        \item Designed and implemented a home automation system using NodeMCU and the Blynk mobile application for wireless control.
    \end{itemize}
    \projectItem[
            title={IoT Based Smart Warehouse Monitoring System},
            duration={Sept 2023 – Jan 2024},
            keyHighlight={Tools \& Technologies: ESP8266, BME680 sensor, Arduino IDE, Blynk App}
        ]
     \begin{itemize}
        \vspace{-0.5em}
        \itemsep -6pt {}
        \item Developed a real-time IoT-based system for monitoring temperature, humidity, pressure, and CO\(_2\) levels using the BME680 sensor and ESP8266.
    \end{itemize}
    \projectItem[
        title={Design and Verification of a High speed CRC-32 Generator for Real time Applications},
        duration={Jan 2025 – May 2025},
        keyHighlight=Simulated and verified the design using Vivado
    ]
    \begin{itemize}
        \vspace{-0.5em}
        \itemsep -6pt {}
        \item Designed and implemented a 32-bit parallel Cyclic Redundancy Check (CRC) architecture capable of processing 64-bit input data using syndrome table-based approach.
    \end{itemize}

    \projectItem[
        title=Universal Asynchronous Receiver/Transmitter Implementation on FPGA,
        duration=Jan 2025 – May 2025,
        keyHighlight=UART is implemented using the VIVADO for Basys3 FPGA board]\\
    \projectItem[
            title={Simulation of CMOS Potentiostat Circuit for Electrochemical Sensors},
            duration={Aug 2024 – Jan 2025},
            keyHighlight={Designed and simulated a CMOS potentiostat circuit optimized for electrochemical sensors using Cadence Design Systems, Generic 90nm PDK (gpdk090)}
        ]
\end{experienceSection}
% --------- Education ---------
\begin{educationSection}{Education}
    \educationItem[
        university={Kerala University of Digital Sciences, Innovation and Technology},
        college={(Digital University Kerala), Thiruvananthapuram},
        graduation={2025},
        grade={7.31/10 GPA},
        program={MSc Applied Physics with specialization in VLSI Design },
        coursework={Digital Chip Deign and Verification, Mixed Signal VLSI Physical Design Lab, CMOS Integrated Operational Amplifier, Verilog Programming and Deep Learning Lab, Neuromorphic VLSI, Data Converters} ,
    ]\\   
    \educationItem[
    university={University of Calicut},
    college={Pookoya Thangal Memorial Government College, Perinthalmanna},
    graduation={April,2020},
    grade={2.88/6 CGPA},
    program={BSc Physics},
    coursework={Electronics, Electrodynamics, Solid state physic spectroscopy and laser physics}
    ]
\end{educationSection}

\end{document}
